\documentclass{article}[11pt]
\usepackage[margin=1.5in]{geometry}
\usepackage{amssymb}
\usepackage{epsfig}
\usepackage{subfigure}
\usepackage{graphicx}
\usepackage{textcomp}
\usepackage{cite}
\usepackage{float}
\usepackage{csquotes}
\makeatletter
\makeatother

\expandafter\let\csname equation*\endcsname\relax
\expandafter\let\csname endequation*\endcsname\relax
\usepackage{amsmath}
\newcommand{\agt}{\mathrel{\raise.3ex\hbox{$>$\kern-.75em\lower1ex\hbox{$\sim$}}}}
\begin{document}

\title{\emph{Frankenstein} and the horrors of competitive exclusion}


\author{N. D. Dominy \& J. D. Yeakel}

% \begin{abstract}
%
%
% \end{abstract}


\maketitle

Bicentennial celebration of the inception of Frankenstein motivates the present comment on Victor Frankenstein and his fateful decision to destroy an unfinished female creature. The act was impulsive (caused by a``sensation of madness"), but it was preceded with agonized reasoning that would be familiar to any contemporary ecologist or evolutionary biologist. Here we present a formal treatment of this reasoning. Our findings suggest that the central horror of Mary Shelley's lies in its prescient mastery of foundational concepts in ecology and evolution.



A little background: Victor Frankenstein created and disavowed a nameless male creature described as eight feet in height, and proportionally large. For three years, the frightened creature wandered the European wilderness, becoming literate in three languages and well mannered. The creature reunites with Frankenstein in Switzerland and pleads for a female companion ``of the same species" to mitigate his loneliness. Crucially, and cleverly, the creature appears to anticipate and counter concerns of direct competition with humans; he promises dispersal to a region of low population density and emphasizes niche differentiation:


\begin{displayquote}
``If you consent, neither you nor any other human being shall ever see us again: I will go to the vast wilds of South America. My food is not that of man; I do not destroy the lamb and the kid to glut my appetite; acorns and berries afford me sufficient nourishment. My companion will be of the same nature as myself, and will be content with the same fare. We shall make our bed of dried leaves; the sun will shine on us as on man, and will ripen our food."
\end{displayquote}


Frankenstein is persuaded by this argument and he consents to create a female creature. However, he soon considers the potential for population growth and direct competition, ``a race of devils would be propagated upon earth who might make the very existence of the species of man a condition precarious and full of terror." The nature of this terror is expressed more clearly a few lines later, and it appears revolve around the concepts of competitive exclusion and extinction: ``future ages might curse me as their pest, whose selfishness had not hesitated to buy its own peace at the price, perhaps, of the existence of the whole human race."



Such reasoning anticipates foundational concepts in ecology and evolution, and it is worth asking if and when humans would be driven to extinction. segue here...







\end{document}
